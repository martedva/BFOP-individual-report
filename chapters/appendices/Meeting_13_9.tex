This was the first meeting with Aasted. The project was in the initial phase. The main focus of this meeting was to find problem areas in other to define the initial scope of the project.\\\\ 
\textbf{The participants of the meetings were:}\\
The SSL group\\
Piet\footnote{Piet Tæstesen, CEO}\\
Mia\footnote{Mia Mortensen, Piet's personal assistant and SSL's contact person}\\\\
\textbf{About Aasted}\\
Aasted er en nicheforretning, hvor de ligger top 3 globalt. Markedet er påvirket af et massivt opkøbstogt fra investorer, som gerne vil investere i industri, i stedet for at smide pengene i banken.\\
Aasted er ”high end”, da de har højere priser end konkurrenterne. Til gengæld laver de skræddersyet løsninger (black box) i stedet for billigt salg af færdige løsninger. Aasted er en af de mindre spillere, som vinder på deres udvikling, innovations og patent niche.\\
Det meste af Aasteds IT-struktur bygger på projekter, hvilket deres salg også gør (ERP, Enterprise Resource Planning).\\
Aasted har primært store producenter som kunder, men forsøger at fokusere nedad i deres segmentering af kunder.\\
Aasted har (næsten) ingen produktion (kun temper). Delene kommer udefra gennem sourcing, og Aasted står for samling heraf (montage/assembly).\\\\
\textbf{Business Model Canvas}\\
Key Partners:\\
- AAK Aarhuskarlshamn UK Ltd: Samarbejdspartner indenfor udvikling og forskning. Specialister inden for enzymer. Samarbejde går ud på maskiner fra Aasted mod afprøvning og forskning fra AAK.\\
- Universitetspartnere (CBS, DTU): Primært forskningssamarbejde.\\
- Virksomhed i Ukraine (Aasted ejer 15-20\%): Produktions offshoring.\\
\textbf{Key activities:}\\
- Sourcing af dele til montage af maskiner og ”fabrikker”.\\
\textbf{Revenue Stream:}\\
- Produktgrupper af dækningsbidrag betyder stor forskel i salg heraf. Margin for indtjeningen spænder fra 35-70\%.\\
- Reservedele hører under udstyr i forbindelse med ”revenue streams”.\\
- Aasted Academy (nyt) koncept – oplæring af virksomheder i udstyr og maskiner, hvorefter de tager disse med tilbage.\\\\
\textbf{New Company Strategy}\\
1. Bageri – Øget fokus på bageri, da det vækster mere end chokolade (større vækstrate). Dette er især i forbindelse med det geografiske område øst for Indien, hvor det er billigere at producere og opbevare.\\
2. Kundesegmentering – Opdeling i top 100, top 1.000, top 10.000 (virksomhedsstørrelse).

a. Top 100 - Aasted har lavet aftaler med 80-85 af de 100 største inden for de sidste 24 år. Dette er typisk større koncerner, hvor salget er baseret på deres opbyggede relation.

b. Top 1.000 – Den nye strategi har større fokus på dette segment. De har andre behov, da de køber anderledes end top 100. Top 1.000 er ofte privatejede, som tillader et øget fokus på deres værdi omkring personligt salg. Dette er endda til tider fra familie til familie, hvor handlen sker i egen stue.

Til dette er der dog brug for nye salgsprocesser, analyse af kunders købsprocesser samt, vigtigst af alt, opbygning af helt nye relationer.\\
3. Geografisk: Sats på flere land mod øst – her er tale om Indien, Pakistan, osv. Dette punkt bunder ud i den øgede fokus på bageri.\\
Målet for denne strategi er at skabe et nyt fundament, som kan bære den planlagte vækst. Aasted har lige nu et fundament, som bygger på, at de ansatte stadig ser virksomheden, som en mindre virksomhed, end den i virkeligheden er.\\\\
\textbf{Pains}\\
\textbf{1. Planlægning og budgettering af projekter i Aasted}\\
Ansvarlig – Kjeld Jørgensen og Niels.\\
Fokusområde – Dette problem omhandler hele flowet i salg af større løsninger.\\
Afdeling – Alle (overordnet service, eller projektleder).\\
ERP:\\
- Oprettelse af ordre sker her. Et ordrenummer bliver tildelt. 5044 (unikt kundenummer) – 02 (maskintype) – 01 (antal).\\
- Programmet sætter ordren op i styklister og produktioner.\\
- ERP opretter automatisk ordren rundt i virksomheden og sørger for alle bliver noteret.\\
- Projektlederen overtager herefter projektet de næste ca. 12 måneder, hvor projektstyring sker i Excel.\\
Excel:\\
- Projektet oprettes i sektioner alt efter afdeling. Projektstyring sker bl.a. gennem Gantt diagrammer.

o Tingene går sjældent efter planen.

o Iterationer styres her, men bliver ikke gjort, da det er for uoverskueligt.\\
- Excel bruges som en manuel tjekliste og oversigt over projektet. Eventuelt færdige eller ændrede arbejdsopgaver rettes her.\\
- Præudfyldt data heri er et stort problem, da projekter typisk er 12 måneder, og projektlederen fx. ikke aner, hvornår ting ankommer, så projektet kan fortsætte.\\
Foreslået løsning – Planningboard, et IT-værktøj, som er købt af Aasted. Dette værktøj er integreret med deres ERP og bliver brugt af montørerne. Denne løsning blev foreslået som løsning til at erstatte Excel for projektlederen, om muligt.\\\\
\textbf{2. Logging af kundekontakt og intern kommunikation}\\
Ansvarlig – Jan Bruun\\
Fokusområde – Dette problem omhandler hele flowet i forbindelse med kundens kontakt med service og den interne kommunikation i denne forbindelse.\\
Afdeling – Service\\
- Kontakt til service sker ved at ringe ind, maile ind, via hjemmesiden eller sociale medier. Der er typisk tale om kunder, agenter (sælgere der ikke er direkte ansat af Aasted) eller montører.\\
- Der er brug for et ”ticket system”, som skaber overblik og statistik over kommunikationskanaler samt deres videre interaktion med firmaet – altså den interne kommunikation.\\
- Der er både tale om tracking af kanaler, men også logging heraf samt analyse af dette til statistik.\\
\hrule