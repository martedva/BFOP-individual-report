This meeting took place after the data gathering of the in-depth phase. The main purpose for this meeting was to present the findings for the steering committee, and to pinpoint potential misunderstandings. It was the goal that the innovation phase could begin shortly after this meeting.\\\\ 
\textbf{The participants of the meetings were:}\\
The SSL group\\
Morten\footnote{Morten Pilnov, CCO and main decision maker of the steering committee}\\
Christian\footnote{Christian Ingemann, Head of Service}
\\\\
\textbf{General notes}\\
Der er 56 åbne garantisager i dag (3/11-2017). 551 i år.\\\\
Vigtigt for endeligt system:\\
Hvordan dækker systemet andre sager end garantisager?\\
Skal være fleksibelt - det er vigtigt, at det kan kobles sammen med andre systemer, da de i forvejen har for mange systemer.\\\\
Aasted investerer ud fra profit. Hvis de ikke tjener nok, kan planer ændre sig.\\\\
\textbf{Vigtig:}\\ 
Jeg (Jannik) tror, at det er en rigtig god idé, hvis vi ud over et system foreslår nogle processændringer, der kan begyndes på med det samme.\\
Som Jens siger: Hvis folk bare kendte processen, ville det nok hjælpe allerede.\\\\
\textbf{Problemformulering}\\\\
\textit{Succeskriterier:}\\\\
1. “Be able to get an overview of all non-completed cases.”\\
Ikke kun non-completed cases. Man skal kunne se historisk data.\\\\
2. “Might help a service employee...”\\
Fjern ordet might. Antag, at hvis information er indsamlet, så VIL det hjælpe at slå det op.\\\\
3. “Get an overview of a case within xx minutes/hours...” (5 minutter, under opkald)\\
Aasted har ikke nogen historisk data. Opslag skal være mulig, mens man taler med kunden.\\\\
4. “Rate at which a service employee can pick up a case transferred…”\\
Det burde omformuleres til, at det ændrer noget, at en person er syg. Man skal kunne oprette eller finde en case, idet man snakker i telefon med kunden.\\
Det er vigtigt, at man kan oprette en sag mens man taler med kunden.\\\\
5. Ingen kommentarer\\\\
6. “Ensure that no case is postponed indefinitely due to having a low priority.”\\
Giv evt. hver sag en timer, der indikerer, hvornår det ikke kan vente længere. Denne timer kan være differentieret ift. forskellige kategorier af sager.\\\\
Tilføj evt.  et ekstra succeskriterie. Man skal kunne trække data ud af et evt. system, som kan bruges i andre afdelinger/systemer.\\\\
\textbf{Workflow}\\
\textbf{Vigtigt:} Aasted vil gerne beholde følelsen af familievirksomhed. Det betyder også, at management/direktion/ejere, skal kunne se kundedata.\\\\
Ville være godt, hvis man kan samle data fra gamle systemer. Inden 2020, vil DAX formentlig skulle udskiftes. \textbf{Vi må meget gerne undersøge, om servicering af DAX stopper snart.}\\\\
\textbf{Vigtigt:}\\
Varer strander ofte i tolden. Det er kun shipping, der har den information.
\\\\
\textbf{Future System}\\
Spørgsmål:\\
Hvilke resourcer har Aasted til rådighed?\\
Er det et helt nyt system?\\ 
Haster det?\\ 
Er det et 2020 projekt?
\\\\
Citat af Morten:
\begin{quote}
"Det her det haster lidt. Det er første halvdel af 2018, og jeg vil sige, at få processen dokumenteret og få den italesat før systemet, det er vigtigst faktisk. Bare ud fra det, i har sagt lige nu, der kunne man på hver af de brandpunkter der var - man kunne vælge de fem største brandpunkter - og så kunne man lave fem initiativer på hver, som man kunne smide direkte ind, og så kunne man begynde at arbejde på det og så lave nogle lommeløsninger. Så jeg synes processen for udviklingen, den skal vi i gang med. Systemmæssigt, så vil jeg godt nøjes, indtil der er noget andet, der gør. Så man ville nok ikke gå ud og købe den forkromede løsning med det samme, det kunne jeg ikke forestille mig vi ville gøre i 2018, men om man ville lave en eller anden form for sharepoint, workflow i det her, hvis det kun var garantidelen af det, eller om man ville få en til at programmere noget i excel."
\end{quote}
Samtale om cloud med Morten:
\begin{quote}
	Morten: "Jeg har en kone der har været 10 år hos IBM der er hos Oracle nu, så jeg er lidt farvet af cloud-solutions og hvad man nu kan af løsninger den vej rundt."
	\\
	(...)
	\\
	Jens: "Så vidt jeg kan forstå på Erik (IT-chef) så er det egentlig ikke IT (afdelingen) der bremser det, det er fra toppen man har taget beslutningen"
	\\
	Morten: "Det er fordi man ikke forstår det. Det har ikke noget med dét at gøre, man forstår det bare ikke."
\end{quote}