\textbf{Business Model Canvas}, BMC, constitutes the foundation for the cooperation and business case. By examining \textit{Aasted ApS}, mainly through their website\footnote{http://wwww.aasted.eu}, we were able to complete our BMC. In our project it was of vital importance, as it uncovered the company's value and purpose prior to our first meeting.
\\\\
The BMC served as a guideline for our first meeting with \textit{Aasted ApS} and their CEO, \textit{Piet Tæstensen}. This allowed us to, at our very first meeting, uncover the strategies, goals and IT circumstances of the organization as well as their current pains. With the help of some provided documentation, this meant that because of the BMC, we were able to cover most of the intiation and in-line phase in one meeting.
\\\\
\textbf{Interviews} was the primary MUST technique in our group, conducted at \textit{Aasted ApS} in the employees respective department. The interviews were all conducted in the in-depth phase for us to understand work processes, task, work functions, and issues regarding these. This gave us concrete experience with interviewees work practices, IT usage and technological options (the knowledge areas D-F\footnote{Bibtex PID p. 200}).
\\\\
Immediately after an interview, we compared notes, consulting the recorded audio, if any disagreements arose concerning the context. Following the comparison a summary was written, checking the recorded audio for any missed points. Lastly the summary was forwarded to the interviewee to confirm, comment and correct the content. This gave us concrete and confirmed data, which proved to be very helpful arguments in the form of diagnostic maps.
\\\\
\textbf{Diagnostic Maps}
was used to identify issues, their cause, and their consequence. We produced Diagnostic Maps for all interviews conducted in the Service department immediately after writing the summaries. At the end of the in-depth phase the mapping was then used to relate problems to our ideas for solutions. This constituted useful argumentation for the relevance of our proposed visions with the steering committee, effectively enforcing the  MUST principle anchoring visions\footnote{Bibtex PID p. 212}.
\\\\
\textbf{Observations}  complimented our interviews with first-hand experience of work practices, effectively helping us avoid the \textit{say-do problem}\footnote{PID p. 246}. A \textit{passive observation}\footnote{Bib-tex PID p. 210} was explicitly coordinated with the observed employees to avoid \textit{thinking-aloud}\footnote{Bib-tex PID p. 210} behavior, which would hinder the detection of any \textit{say-do problems}. The observation was represented in a workflow diagram for an easier, more visual overview, than a summary - the proposed representation tool.