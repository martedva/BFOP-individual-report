Forklar: Hvorfor valgte vi det givne værktøj? Hvordan hjalp det os?\\\\
\textbf{Business Model Canvas}
BMC\\
intial anchoring.
\\\\
\textbf{Interviews} was the primary MUST technique in our group, conducted at \textit{Aasted ApS} in the employees respective department. The interviews were all conducted in the in-depth phase to understand work processes, task, work functions, and issues regarding these. This gave us concrete experience with \footnote{Bibtex PID p. 200} interviewees work practices, IT usage and technological options (the knowledge areas D-F).
Initiation, in-line.\\
All points covered.
\\\\
\textbf{Diagnostic Maps}
was used to identify issues, their cause, and their consequence, we produced Diagnostic Maps for all interviews conducted in the Service department immediately after the interview. At the end of the in-depth phase the mapping was then used to relate problems to our ideas for solutions. This constituted useful argumentation for the relevance of our proposed visions with the steering committee, effectively enforcing the  MUST principle anchoring visions\footnote{Bibtex PID p. 212}.
\\\\
\textbf{Observations}  complimented our interviews, observations were used to gain a first-hand experience of work practices, effectively avoiding the \textit{say-do problem}\footnote{PID p. 246}. A \textit{passive observation}\footnote{Bib-tex PID p. 210} was explicitly coordinated with the observed employees to avoid \textit{thinking-aloud}\footnote{Bib-tex PID p. 210} behaviour, which would hinder the detection of any \textit{say-do problems}. The observation was represented in a workflow diagram for easier, more visual overview, than a summary - the proposed representation tool.
Say-do avoided.\\
Thinking aloud, kvante\\
\\\\
\textbf{Presentation of Solutions}
Vital feedback.
\\\\
Review?